\documentclass[10pt,fleqn]{article}

\usepackage{graphicx}
\usepackage[tableposition=top]{caption}
\usepackage{natbib}
\usepackage{floatrow}
\floatsetup[table]{capposition=top}

\usepackage[]{mcode}

\usepackage{fancyhdr}
% \usepackage[margin=1in,showframe]{geometry}
\usepackage[margin=1in]{geometry}
\usepackage{tabularx} % for \textwidth table
\usepackage{multicol}
\usepackage{hyperref}
\hypersetup{
  colorlinks,
  urlcolor=blue,
  citecolor=gray,
  pdftitle={DM Curriculum Vitae},
  pdfauthor={Dylan Mikesell},
  pdfsubject={Curriculum Vitae},
}

\pagestyle{fancy}

\lhead{Boise State University}
\chead{GEOS 397}
\rhead{Fall 2016}
\fancyfoot[C]{\thepage}
% \renewcommand{\headrulewidth}{0pt} % remove horizontal line
\setlength{\headsep}{12pt} % set separation between course name and header
% \setlength{\hoffset}{-0.5in} % set separation between course name and header
% \setlength{\textwidth}{7in}
\setlength\parindent{0pt} % Removes all indentation from paragraphs

\title{Computation in the geosciences}

\date{\empty}

\setlength{\parindent}{0pt}
\setlength{\parskip}{\baselineskip}


\usepackage{siunitx}
\sisetup{output-exponent-marker=\ensuremath{\mathrm{e}}}

% \renewcommand{\chaptertitle}[1]{\chaptitlefont\MakeUppercase{#1}}

\begin{document}

% make the title
% \begin{multicols}{2}
% \begin{flushleft}
\textbf{L16: class exercise} -- Try to implement/solve the following problems in MATLAB.
% \end{flushleft}
% \columnbreak 
% \begin{flushright}
% \textbf{Due: 5:00 PM 10/17/16}
% \end{flushright}
% \end{multicols}


\section*{Numerical Integration}

Numerically integrate the following using \textit{quad} or \textit{integral}:

\begin{eqnarray} \nonumber
&& \int\limits_0^{3} \sqrt{y+1} dy \\ \nonumber
&& \int\limits_{-1}^{1} \frac{5r}{(4+r^2)^2} dr  \\ \nonumber
&& \int\limits_0^{\pi/6} \cos^{-3}(2\theta) \sin(2\theta) d\theta \\ \nonumber
&& \int\limits_0^{\pi/2} e^{\sin(x)}\cos(x) dx \\ \nonumber
&& \int\limits_0^{\sqrt{ln\pi}} 2xe^{x^2}\cos(e^{x^2}) dx  \\ \nonumber
&& \int\limits_1^{4} \frac{dy}{2\sqrt{y}(1+\sqrt{y})^2}  \\ \nonumber
\end{eqnarray}

\section*{Compare a simple function and discrete data}

\begin{enumerate}
	\item Plot the function $y(x)=xe^{-x}$ between $x=0$ and $x=5$ using 101 points.
	\item Using integration by parts, show that the exact value of $I(a) = \int\limits_0^a y(x) dx = \int \limits_0^a xe^{-x} dx = 1-e^{-a} - ae^{-a}$.
	\item Write a \textbf{function} that computes the exact area of $A$ using the solution in the previous step. Call this function \textit{exactArea.m}.
	\item Use your function to compute the value of $I(a=5)$.
	\item Use matlab's built in \textit{trapz} function to compute the approximate integral of your discrete function in step 1.
	\item Repeat the \textit{trapz} integration when the number of points is 11 and 1001. 
	\item Compare the differences between the exact solution to the integral and the approximate solutions when you use 11, 101 and 1001 points to compute the integral of $y(x)$ from 0 to 5.
	\item Repeat steps 5, 6 and 7 using \textit{quad} or \textit{integral} instead of \textit{trapz}.
\end{enumerate}

\end{document}
