\documentclass[10pt,fleqn]{article}

\usepackage{graphicx}
\usepackage[tableposition=top]{caption}
\usepackage{natbib}
\usepackage{floatrow}
\floatsetup[table]{capposition=top}

\usepackage[]{mcode}

\usepackage{fancyhdr}
% \usepackage[margin=1in,showframe]{geometry}
\usepackage[margin=1in]{geometry}
\usepackage{tabularx} % for \textwidth table
\usepackage{multicol}
\usepackage{hyperref}
\hypersetup{
  colorlinks,
  urlcolor=blue,
  citecolor=gray,
  pdftitle={DM Curriculum Vitae},
  pdfauthor={Dylan Mikesell},
  pdfsubject={Curriculum Vitae},
}

\pagestyle{fancy}

\lhead{Boise State University}
\chead{GEOS 397}
\rhead{Fall 2016}
\fancyfoot[C]{\thepage}
% \renewcommand{\headrulewidth}{0pt} % remove horizontal line
\setlength{\headsep}{12pt} % set separation between course name and header
% \setlength{\hoffset}{-0.5in} % set separation between course name and header
% \setlength{\textwidth}{7in}
\setlength\parindent{0pt} % Removes all indentation from paragraphs

\title{Computation in the geosciences}

\date{\empty}

\setlength{\parindent}{0pt}
\setlength{\parskip}{\baselineskip}


\usepackage{siunitx}
\sisetup{output-exponent-marker=\ensuremath{\mathrm{e}}}

% \renewcommand{\chaptertitle}[1]{\chaptitlefont\MakeUppercase{#1}}

\begin{document}

\textbf{L20: class exercises} -- Try to implement/solve the following problems in MATLAB.

\section*{Correlation}

Write your own correlation coefficient computation using the equation

$$r = \frac{n \sum xy - (\sum x)(\sum y)}{\sqrt{n(\sum x^2)-(\sum x)^2}\sqrt{n(\sum y^2)-(\sum y)^2}}.$$

Write a function to compute this correlation coefficient given two inputs $x$ and $y$.

Test this function with the following:

\begin{itemize}
	\item Make an x-vector from -10:10
	\item Compute the function $y(x) = 20x - 5$
	\item Add random Gaussian noise to this y-vector with an amplitude of 10
	\item Use \textbf{polyfit()} to to estimate the coefficients of the first order polynomial 
	\item Compute the correlation coefficient between $x$ and your noisy $y$
	\item Compute the correlation coefficient between $x$ and your predicted $y$ after you estimate the polynomial coefficients
\end{itemize}

Now that you have an estimate of the polynomial coefficients and have predicted a new y-vector based on these coefficients, we can look at the properties of the noise we added.

\begin{itemize}
	\item Subtract the estimated y-vector from the noisy y-vector. Call this new vector \textit{yDiff}.
	\item Plot this new vector.
	\item Compute the mean of this vector. What is it?
	\item Compute the standard deviation. What is it?
	\item Try increasing the number of points in the x-vector and recomputing the above values. What do you notice about the standard deviation and the mean?
\end{itemize}


\end{document}






