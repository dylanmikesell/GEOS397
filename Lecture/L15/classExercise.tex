\documentclass[10pt,fleqn]{article}

\usepackage{graphicx}
\usepackage[tableposition=top]{caption}
\usepackage{natbib}
\usepackage{floatrow}
\floatsetup[table]{capposition=top}

\usepackage[]{mcode}

\usepackage{fancyhdr}
% \usepackage[margin=1in,showframe]{geometry}
\usepackage[margin=1in]{geometry}
\usepackage{tabularx} % for \textwidth table
\usepackage{multicol}
\usepackage{hyperref}
\hypersetup{
  colorlinks,
  urlcolor=blue,
  citecolor=gray,
  pdftitle={DM Curriculum Vitae},
  pdfauthor={Dylan Mikesell},
  pdfsubject={Curriculum Vitae},
}

\pagestyle{fancy}

\lhead{Boise State University}
\chead{GEOS 397}
\rhead{Fall 2016}
\fancyfoot[C]{\thepage}
% \renewcommand{\headrulewidth}{0pt} % remove horizontal line
\setlength{\headsep}{12pt} % set separation between course name and header
% \setlength{\hoffset}{-0.5in} % set separation between course name and header
% \setlength{\textwidth}{7in}
\setlength\parindent{0pt} % Removes all indentation from paragraphs

\title{Computation in the geosciences}

\date{\empty}

\setlength{\parindent}{0pt}
\setlength{\parskip}{\baselineskip}


\usepackage{siunitx}
\sisetup{output-exponent-marker=\ensuremath{\mathrm{e}}}

% \renewcommand{\chaptertitle}[1]{\chaptitlefont\MakeUppercase{#1}}

\begin{document}

% make the title
\begin{multicols}{2}
% \begin{flushleft}
\textbf{L15: class exercise}
% \end{flushleft}
% \columnbreak 
% \begin{flushright}
% \textbf{Due: 5:00 PM 10/17/16}
% \end{flushright}
\end{multicols}

Try to implement/solve the following problems in MATLAB.


\section*{Plotting}

1) Use the MATLAB \textbf{peaks()} function to generate X,Y and F meshes that have dimensions 40x40.

2) Plot the data as a surface (i.e. use \textbf{surf()}) with the colormap 'jet'. Add a colorbar with a color label. Use the \textbf{hold on} command and use \textbf{imagesc()} to plot the matrix image.

\textit{Question: What kind of plot does this make. Desribe the plot.}

3) Add 10 to the function F and redo the steps in part 2). 

\textit{Question: What does the image now look like?}

\textit{Question: What is adding 10 to the function F actually doing to the plot?}


\section*{Numerical Integration}

Numerically integrate the following:

\begin{eqnarray} \nonumber
&& \int\limits_0^{3} \sqrt{y+1} dy \\ \nonumber
&& \int\limits_{-1}^{1} \frac{5r}{(4+r^2)^2} dr  \\ \nonumber
&& \int\limits_0^{\pi/6} \cos^{-3}(2\theta) \sin(2\theta) d\theta \\ \nonumber
&& \int\limits_0^{\pi/2} e^{\sin(x)}\cos(x) dx \\ \nonumber
&& \int\limits_0^{\sqrt{ln\pi}} 2xe^{x^2}\cos(e^{x^2}) dx  \\ \nonumber
&& \int\limits_1^{4} \frac{dy}{2\sqrt{y}(1+\sqrt{y})^2}  \\ \nonumber
\end{eqnarray}

\end{document}
