\documentclass[10pt,fleqn]{article}

\usepackage{graphicx}
\usepackage[tableposition=top]{caption}
\usepackage{natbib}
\usepackage{floatrow}
\floatsetup[table]{capposition=top}

\usepackage[]{mcode}

\usepackage{fancyhdr}
% \usepackage[margin=1in,showframe]{geometry}
\usepackage[margin=1in]{geometry}
\usepackage{tabularx} % for \textwidth table
\usepackage{multicol}
\usepackage{hyperref}
\hypersetup{
  colorlinks,
  urlcolor=blue,
  citecolor=gray,
  pdftitle={DM Curriculum Vitae},
  pdfauthor={Dylan Mikesell},
  pdfsubject={Curriculum Vitae},
}

\pagestyle{fancy}

\lhead{Boise State University}
\chead{GEOS 397}
\rhead{Fall 2016}
\fancyfoot[C]{\thepage}
% \renewcommand{\headrulewidth}{0pt} % remove horizontal line
\setlength{\headsep}{12pt} % set separation between course name and header
% \setlength{\hoffset}{-0.5in} % set separation between course name and header
% \setlength{\textwidth}{7in}
\setlength\parindent{0pt} % Removes all indentation from paragraphs

\title{Computation in the geosciences}

\date{\empty}

\setlength{\parindent}{0pt}
\setlength{\parskip}{\baselineskip}


\usepackage{siunitx}
\sisetup{output-exponent-marker=\ensuremath{\mathrm{e}}}

% \renewcommand{\chaptertitle}[1]{\chaptitlefont\MakeUppercase{#1}}

\begin{document}

\textbf{L21: class exercises} -- Try to implement/solve the following problems in MATLAB.

\section*{Linear interpolation}

Load the two data sets \textit{series1.txt} and \textit{series2.txt} using the \textbf{load()} function. Both synthetic data sets contain a two-column matrix with 339 rows. These data come from the book \textit{MATLAB Recipes for Earth Sciences}.

The first column contains ages in kiloyears, which are unevenly spaced. The second column contains oxygen-isotope values measured on calcareous micro-fossils (foraminifera). The data sets contain 100, 40 and 20 kyr cyclicities and they are overlain by Gaussian noise. In the 100~kyr frequency band, the second data series (series 2) is shifted by 5 kyrs with respect to the first data series (series 1).

\begin{enumerate}
	\item Make a new time vector from 0 to 996~kyr, with a sample interval of 3~kyr.
	\item Use \textbf{interp1()} to linearly interpolate both series1 and series2. \textit{(You may need to look at the MATLAB help/doc for interp1.)}
	\item In one figure, use subplot 1 to plot the raw data and interpolated data. Window the x-axis from 350 to 450~kyr. \textit{(It may help to the plot the raw data as points and the interpolated data as a line.)}
	\item In the same figure on subplot 2, plot the same thing but for series 2.
\end{enumerate}

If time, try redoing this exercise, using one of the other interpolation methods instead of 'linear' interpolation. Do you notice any differences?

\begin{itemize}
  \item Piecewise Linear Interpolation (linear)
  \item Piecewise Cubic Hermite Interpolation (cubic)
  \item Shape-Preserving Piecewise Cubic Interpolation (pchip)
  \item Piecewise Cubic Spline Interpolation (spline)
\end{itemize}


\end{document}






