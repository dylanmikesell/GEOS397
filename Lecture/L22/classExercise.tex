\documentclass[10pt,fleqn]{article}

\usepackage{graphicx}
\usepackage[tableposition=top]{caption}
\usepackage{natbib}
\usepackage{floatrow}
\floatsetup[table]{capposition=top}

\usepackage[]{mcode}

\usepackage{fancyhdr}
% \usepackage[margin=1in,showframe]{geometry}
\usepackage[margin=1in]{geometry}
\usepackage{tabularx} % for \textwidth table
\usepackage{multicol}
\usepackage{hyperref}
\hypersetup{
  colorlinks,
  urlcolor=blue,
  citecolor=gray,
  pdftitle={DM Curriculum Vitae},
  pdfauthor={Dylan Mikesell},
  pdfsubject={Curriculum Vitae},
}

\pagestyle{fancy}

\lhead{Boise State University}
\chead{GEOS 397}
\rhead{Fall 2016}
\fancyfoot[C]{\thepage}
% \renewcommand{\headrulewidth}{0pt} % remove horizontal line
\setlength{\headsep}{12pt} % set separation between course name and header
% \setlength{\hoffset}{-0.5in} % set separation between course name and header
% \setlength{\textwidth}{7in}
\setlength\parindent{0pt} % Removes all indentation from paragraphs

\title{Computation in the geosciences}

\date{\empty}

\setlength{\parindent}{0pt}
\setlength{\parskip}{\baselineskip}


\usepackage{siunitx}
\sisetup{output-exponent-marker=\ensuremath{\mathrm{e}}}

% \renewcommand{\chaptertitle}[1]{\chaptitlefont\MakeUppercase{#1}}

\begin{document}

\textbf{L22: class exercises} -- Try to implement/solve the following problems in MATLAB.

\section*{Time series}

In this step you will create your own time series that contains multiple periodic components, a linear trend and noise. Create a time vector from 0:1000 with a time step of dt=1.

Now create a periodic signal ($x_p$) composed of three sine functions with 1) A1=2, f1=1/50; 2) A2=1, f2=1/15, and 3) A3=0.5, f3=1/5 summed together. For example, sine wave one would be A1*sin(2*pi*f1*t). Plot this wave and label the axes.

Next, create a linear trend ($x_tr$) that has a slope of 0.005 and y-intercept at 0. Add this to the $x_p$ curve and plot on top of the individual $x_p$ curve. Do you notice a linear trend?

Finally, generate zero-mean random Gaussian noise with \textbf{randn()}. The noise should have a standard deviation of one. \textit{Note: Prior to calling \textbf{randn()}, make sure to set \textbf{rng(0)} so that everyone in class has the same random noise.} Add this noise to the periodic and trend data so that you have a time series with all three components. Plot this data on top of the periodic only signal.

You should now have three signals.
\begin{enumerate}
  \item $x_p(t)$ = periodic signal composed of three sine curves
  \item $x_1(t) = x_p(t) + x_{tr}(t)$ = periodic plus linear trend
  \item $x_2(t) = x_p(t) + x_{tr}(t) + x_n(t)$ = periodic plus linear trend plus noise
\end{enumerate}

\section{Periodogram}

Use \textbf{nextpow()} to find the next power of two of the number of data points. 
Set \textbf{nfft=$2^{nextpow(N)}$}, where N is the number of data. Then call \textbf{[Pxx,f]=periodogram(x\_p,[],nfft,fs)}, where \textit{nfft} is the number of points in the fft (an \textit{even} number of points makes the fft symmetric and using powers of two enables this to be done easily). $x_p$ is the periodic only signal; \textit{fs} is the sampling frequency 1/dt. Then plot the output with the y-limits set from 0:1000;

\begin{lstlisting}
nfft = 2^nextpow2(N);
[Pxx,f] = periodogram(x_p,[],nfft,fs);

figure;
plot(f,Pxx); grid on;
xlabel('Frequency [Hz]' ylim([0 1000]););
ylabel('Power');
title('Auto-Spectrum');
ylim([0 1000]);
\end{lstlisting}

Describe the output. What do you notice about the upper frequency limit?

Repeat the periodogram computation and plotting using 1) the periodic + trend signal and 2) the periodic, trend and noise signal. Describe the differences in the plots.

Now vary the noise from a standard deviation of 1 to 2 to 3 to 4. What happens to you plots?

\end{document}






