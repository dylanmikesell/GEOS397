\documentclass[10pt,fleqn]{article}

\usepackage{graphicx}
\usepackage[tableposition=top]{caption}
\usepackage{natbib}
\usepackage{floatrow}
\floatsetup[table]{capposition=top}

\usepackage[]{mcode}

\usepackage{fancyhdr}
% \usepackage[margin=1in,showframe]{geometry}
\usepackage[margin=1in]{geometry}
\usepackage{tabularx} % for \textwidth table
\usepackage{multicol}
\usepackage{hyperref}
\hypersetup{
  colorlinks,
  urlcolor=blue,
  citecolor=gray,
  pdftitle={DM Curriculum Vitae},
  pdfauthor={Dylan Mikesell},
  pdfsubject={Curriculum Vitae},
}

\usepackage[yyyymmdd,hhmmss]{datetime}

\pagestyle{fancy}
% \rfoot{Created on \today\ at \currenttime}
\rfoot{\today}
\cfoot{}
\lfoot{Page \thepage}


\lhead{Boise State University}
\chead{GEOS 397}
\rhead{Fall 2016}
% \fancyfoot[C]{\thepage} % old way before adding compile time
% \renewcommand{\headrulewidth}{0pt} % remove horizontal line
\setlength{\headsep}{12pt} % set separation between course name and header
% \setlength{\hoffset}{-0.5in} % set separation between course name and header
% \setlength{\textwidth}{7in}
\setlength\parindent{0pt} % Removes all indentation from paragraphs

\title{Computation in the geosciences}

\date{\empty}

\setlength{\parindent}{0pt}
\setlength{\parskip}{\baselineskip}


\usepackage{siunitx}
\sisetup{output-exponent-marker=\ensuremath{\mathrm{e}}}

% \renewcommand{\chaptertitle}[1]{\chaptitlefont\MakeUppercase{#1}}

\begin{document}

\textbf{L24: class exercises} -- Try to implement/solve the following problems in MATLAB.

\section*{Calculate rake}

Recall that the rake represents the angle between a linear feature and a line defining the strike of a plane.  When the lineation’s trend and plunge are measured, the rake can be determined easily using Cartesian coordinates and the concept of a dot product, or scalar product. 

The first step in the process of estimating the rake is to convert our data from their spherical components into their NED components. You want to use the plunge and trend of the lineation and the trend of the line representing the strike of the plane. \textit{Note: the rake is the angle between lineation and the strike of the plane with zero dip. Therefore, we can represent the plane as a line and instead of using strike and dip, use trend and plunge. In this case, the trend is the strike and the plunge is zero.}

From our notes we know that 
\begin{eqnarray}
  N &=& \cos(plunge)\cos(trend) \\
  E &=& \cos(plunge)\sin(trend) \\
  D &=& \sin(plunge)
\end{eqnarray}
and these three components together make the Cartesian coordinate vector representing this line (i.e. $\vec{u}=[N, E, D]$).

Assuming that we have two vectors ($\vec{u}$ and $\vec{v}$), we can find the angle between those two vectors from the dot product.
\begin{eqnarray} \nonumber
  \vec{u}\cdot\vec{v} &=& |\vec{u}||\vec{v}|\cos(\theta) \\ \nonumber
  \theta &=& \frac{cos^{-1}(\vec{u}\cdot\vec{v})}{|\vec{u}||\vec{v}|}
\end{eqnarray}
Use \textbf{dot()}, \textbf{acosd()} and \textbf{norm()} to compute the rake from the following datasets. For a refresher of the graphical approach to do this, check \href{https://www.uwgb.edu/dutchs/structge/SL85FindPitch.HTM}{this link}. Keep in mind that \textit{pitch} and \textit{rake} are the same thing; different groups of geologists use the different terminology.

\begin{multicols}{3}
\subsection*{Dataset 1}

\begin{lstlisting}
% measured striation data
plunge = 45; % [deg]
trend  = 230; % [deg]
% measured strike of plane
strike = 190; % [deg]
\end{lstlisting}

Answer should be rake=57 deg.

\subsection*{Dataset 2}

\begin{lstlisting}
% measured striation data
plunge = 22; % [deg]
trend  = 320; % [deg]
% measured strike of plane
strike = 300; % [deg]
\end{lstlisting}

Answer should be rake=29 deg.

\subsection*{Dataset 3}

\begin{lstlisting}
% measured striation data
plunge = 30.3; % [deg]
trend  = 182.1; % [deg]
% measured strike of plane
strike = 53; % [deg]
\end{lstlisting}

Answer should be rake=123 deg.
\end{multicols}

\textit{Question: Which part of this process should you automate by writing a function?}

\pagebreak

\section*{Finding the true dip}

To see the graphical approach to this problem look at \href{https://www.uwgb.edu/dutchs/structge/SL83Plane2AD.HTM}{this link}.

Just as we can use calculations in the Cartesian coordinate system to compute rake, we can also compute the true strike and dip of a plane given two observations of the apparent strike and dip of the plane.

The problem boils down to finding the plane that contains two lines.
\textit{Note: The apparent dip is just the dip as seen in a plane oblique to the strike. The strike of the section plane is then the trend of the line and the apparent dip is then the plunge.}

To find the plane that contains the two lines we turn to the cross product. Using the two vectors for our lines, we cannot directly determine the strike and dip of the plane, but we can determine the pole to the plane. Recall that the pole to the plane is normal to the plane and that the cross between two vectors gives you the normal to those two vectors. From the pole of the plane, we can compute the strike and dip.

Given the observation of orientations of two joint faces
\begin{lstlisting}
% line 1;
t1 = 209; % trend 1
p1 = 22; % plunge 1
% line 2
t2 = 56; % trend 2
p2 = 37; % plunge 2
\end{lstlisting}
compute the true strike and dip of the plane containing the joints.

\textbf{Step 1:}
You first need to convert the spherical representation of the data to their Cartesian representation to make a vector for each line.

\textbf{Step 2:}
Use \textbf{cross()} to compute the cross product. Keep in mind that the D-component should be positive down. Therefore, if the D-component is negative, you have two options.
\begin{enumerate}
  \item Multiply the cross product result by -1 to change the direction of the vector to point in the opposite direction
  \item Change the order of the cross product $\rightarrow \vec{u}\times \vec{v} = - \vec{v}\times\vec{u}$. 
\end{enumerate}

\textbf{Step 3:}
Use \textbf{norm()} to make sure that you are dealing with a unit vector for the pole before transforming back to strike and dip.

\textbf{Step 4:}
From the Cartesian vector of the pole, convert back to spherical coordinates to get the plunge and trend of the pole. You should use \textbf{atan2d()} for the trend so that you get the correct quadrant.

\textbf{Step 5:}
Convert from the pole to the plane. To convert from a poles trend and plunge to a planes strike and dip we use the following:
\begin{eqnarray} \nonumber
  strike &=& trend + 90 \\ \nonumber
  dip &=& 90 -  plunge
\end{eqnarray}

\textit{The answer should be strike=38 and dip=68 deg.}


\end{document}
