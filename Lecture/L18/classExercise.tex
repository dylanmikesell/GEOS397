\documentclass[10pt,fleqn]{article}

\usepackage{graphicx}
\usepackage[tableposition=top]{caption}
\usepackage{natbib}
\usepackage{floatrow}
\floatsetup[table]{capposition=top}

\usepackage[]{mcode}

\usepackage{fancyhdr}
% \usepackage[margin=1in,showframe]{geometry}
\usepackage[margin=1in]{geometry}
\usepackage{tabularx} % for \textwidth table
\usepackage{multicol}
\usepackage{hyperref}
\hypersetup{
  colorlinks,
  urlcolor=blue,
  citecolor=gray,
  pdftitle={DM Curriculum Vitae},
  pdfauthor={Dylan Mikesell},
  pdfsubject={Curriculum Vitae},
}

\pagestyle{fancy}

\lhead{Boise State University}
\chead{GEOS 397}
\rhead{Fall 2016}
\fancyfoot[C]{\thepage}
% \renewcommand{\headrulewidth}{0pt} % remove horizontal line
\setlength{\headsep}{12pt} % set separation between course name and header
% \setlength{\hoffset}{-0.5in} % set separation between course name and header
% \setlength{\textwidth}{7in}
\setlength\parindent{0pt} % Removes all indentation from paragraphs

\title{Computation in the geosciences}

\date{\empty}

\setlength{\parindent}{0pt}
\setlength{\parskip}{\baselineskip}


\usepackage{siunitx}
\sisetup{output-exponent-marker=\ensuremath{\mathrm{e}}}

% \renewcommand{\chaptertitle}[1]{\chaptitlefont\MakeUppercase{#1}}

\begin{document}

% make the title
% \begin{multicols}{2}
% \begin{flushleft}
\textbf{L18: class exercises} -- Try to implement/solve the following problems in MATLAB.
% \end{flushleft}
% \columnbreak 
% \begin{flushright}
% \textbf{Due: 5:00 PM 10/17/16}
% \end{flushright}
% \end{multicols}


\section*{Polynomial evaluation}

Evaluate the following polynomial over the range $x=[0,4]$ using \textbf{polyval()} and plot your results on the same plot with a legend for each polynomial; the legend should be the equation of the polynomial. Also in the legend insert the order and name of the polynomial (e.g. order=0, constant equation).

\begin{eqnarray} \nonumber
&& y = 3.75x + 0.25 \\ \nonumber
&& y = 5x + 7 \\ \nonumber
&& y = 2x^2 \\ \nonumber
&& y = 3x^3 + 2x^2 + x\\ \nonumber
&& y = x^7 + 3x - 3 \\ \nonumber
\end{eqnarray}

\section*{Thoughts on polynomials}


\begin{itemize}
	\item Polynomials are easy to deal with in MATLAB
	\item As the order of the polynomial increases so does the complexity of the curve
	\item Remember the Taylor Series?
	\begin{itemize}
		\item You can fit any function with an infinite series of polynomials
		\item More polynomials = better fit
	\end{itemize}
	\item \textbf{polyfit()} is similar to \textbf{polyval} except it fits a polynomial to data rather than creates a polynomial from given coefficients
\end{itemize}

\end{document}









\section*{Compare a simple function and discrete data}

\begin{enumerate}
	\item Plot the function $y(x)=xe^{-x}$ between $x=0$ and $x=5$ using 101 points.
	\item Using integration by parts, show that the exact value of $I(a) = \int\limits_0^a y(x) dx = \int \limits_0^a xe^{-x} dx = 1-e^{-a} - ae^{-a}$.
	\item Write a \textbf{function} that computes the exact area of $A$ using the solution in the previous step. Call this function \textit{exactArea.m}.
	\item Use your function to compute the value of $I(a=5)$.
	\item Use matlab's built in \textit{trapz} function to compute the approximate integral of your discrete function in step 1.
	\item Repeat the \textit{trapz} integration when the number of points is 11 and 1001. 
	\item Compare the differences between the exact solution to the integral and the approximate solutions when you use 11, 101 and 1001 points to compute the integral of $y(x)$ from 0 to 5.
	\item Compute the integral solution using \textit{quad} or \textit{integral} instead of \textit{trapz}. Compare with the exact and the discrete versions of the integral.
\end{enumerate}

\section*{Average (mean) value}

In Lecture~14 we discussed the \textit{area under the curve} and the \textit{average (or mean) value} that can be computed using integration. The formula for the average value of a function is $$av(f) = \frac{1}{b-a} \int \limits_a^b f(x) dx.$$

Using the function $I(a) = \int\limits_0^a y(x) dx$, where $y(x)=xe^{-x}$, to compute the average value of this function analytically.

After you have an equation for the average value, compute the average value when $a=5$.

\section*{Monte Carlo integration}

Sometimes a Monte Carlo based approach is useful. Monte Carlo techniques are quite powerful and involve taking random samples of a function to approximate the average (mean) value of the function. (In a sense, this random sampling can be used as an approximate approach to computing the integral.) In this approach the average value of the function can be computed as $$ av(f) \approx \frac{1}{N} \sum\limits_{i=1}^N y(x_i).$$
But what are $N$ and $x_i$ in this case? Well, $ \{ x_i \}$ is the set of N random samples taken in the integration interval.

In the example above, $x_i$ would be a number within the range of 0 to $a$. You might want to look at this MATLAB \href{https://www.mathworks.com/help/matlab/ref/rand.html}{page} to learn how you would generate the random numbers. Make sure to check the section \textit{Random Numbers Within Specified Interval}.

Write a \textbf{function} to compute the Monte Carlo estimate of the average value of $y(x)=xe^{-x}$. To generate the random samples, use the \textbf{rand()} function. Write your function such that you can input the number of random samples $N$.

Compare your Monte Carlo average value estimate to the exact estimate computed in the previous section. How does the error change with $N$?

From this average value, you can compute the actual value of the integral $I(a)$. Write how you would compute this integral in terms of the Monte Carlo average value.

$$I(a) = \int\limits_0^a y(x) dx = ??$$

In what situations might this approach be useful?

\textit{NOTE: In the discrete data case, you might want to look at this MATLAB \href{https://www.mathworks.com/help/matlab/math/random-integers.html}{page} to learn how you would generate the random sample number instead of random numbers on the x-interval.}

\section*{Example: Averaging air temperature using the trapezoid rule}

An observer measures the outside temperature every hour from noon until midnight, recording the temperatures in the following table:

\begin{table}[bh]
	\centering
	\begin{tabular}{l|ccccccccccccc} \hline
	Time & 0 & 1 & 2 & 3 & 4 & 5 & 6 & 7 & 8 & 9 & 10 & 11 & 12 \\ \hline
	Temp (F) & 63 & 65 & 66 & 68 & 70 & 69 & 68 & 68 & 65 & 64 & 62 & 58 & 55 \\ \hline
	\end{tabular}
	\caption{Air temperature measurements.}
	\label{tab:airtemp}
\end{table}

Using integration, we need to find $$av(f) = \frac{1}{b-a} \int \limits_a^b f(x) dx,$$ but do not know $f(x)$. The integral, however, can be approximated by the \textit{Trapezoid Rule}, using the temperatures in the table as function values at the points of a 12-subinterval partition of the 12-hour interval (i.e. $h=1$).

To approximate the integral use $$\int \limits_a^b f(x) dx \approx \frac{h}{2}\left(y_0 + 2y_1 + 2y_2 + \cdots + 2y_{N-1} + 2y_N \right).$$ The $y's$ are the values of $f(x)$ at the partition points
\begin{eqnarray} \nonumber
x_0 &=& a, \\ \nonumber
x_1 &=& a+h, \\ \nonumber
x_2 &=& a+2h, \\ \nonumber
x_{N-1} &=& a+(N-1)h, \\ \nonumber
x_{N} &=& b,
\label{eqn:partition}
\end{eqnarray}
where $h=(b-a)/N$ and $y_0=f(a)$, $y_1=f(x_1)$, $\ldots$

\textbf{What was the average temperature for the 12-hour period?}
\begin{enumerate}
	\item Use algebra to get the answer.
	\item Implement your own Trapezoid rule to approximate the integral and find the average value.
	\item Use MATLAB's \textbf{trapz()} function to compute the average value.
	\item Does your approximations match the algebraic average?
\end{enumerate}

\vfill

\textit{Examples in this exercise come from 1) Thomas' Calculus: Early Transactions by Finney, Weir and Giodano and 2) Nik Cunniffe's Matlab Practical: Numerical Integration.}

\end{document}
